% Created 2017-11-01 Wed 21:44
% Intended LaTeX compiler: pdflatex
\documentclass[11pt]{article}
\usepackage[utf8]{inputenc}
\usepackage[T1]{fontenc}
\usepackage{graphicx}
\usepackage{grffile}
\usepackage{longtable}
\usepackage{wrapfig}
\usepackage{rotating}
\usepackage[normalem]{ulem}
\usepackage{amsmath}
\usepackage{textcomp}
\usepackage{amssymb}
\usepackage{capt-of}
\usepackage{hyperref}
\date{\today}
\title{}
\hypersetup{
 pdfauthor={},
 pdftitle={},
 pdfkeywords={},
 pdfsubject={},
 pdfcreator={Emacs 25.3.1 (Org mode 9.1)}, 
 pdflang={English}}
\begin{document}

\tableofcontents

\section{Set 1}
\label{sec:org4c8d814}
\subsection{Challenge 1 (Convert hex to base64)}
\label{sec:orgd384c27}
\begin{itemize}
\item Converting hex to base64
\end{itemize}
\begin{verbatim}
from base64 import b64decode, b64encode
import codecs


def hex2b64(s):
    s = codecs.decode(s, "hex")
    return codecs.decode((b64encode(s)), "ascii")


def test_hex2b64():
    s = "49276d206b696c6c696e6720796f757220627261696e206c696b65206120706f69736f6e6f7573206d757368726f6f6d"
    t = "SSdtIGtpbGxpbmcgeW91ciBicmFpbiBsaWtlIGEgcG9pc29ub3VzIG11c2hyb29t"
    return t == hex2b64(s)


test_hex2b64()
\end{verbatim}

\subsection{Challenge 2 (XOR two strings together)}
\label{sec:orgd8ae2eb}
\begin{itemize}
\item Producing the XOR of two fixed-length strings
\item Always operating with bytes, using hex strings only for presentation
\end{itemize}
\begin{verbatim}
import codecs


def xor_strings(xs, ys):
    return bytes([x ^ y for x, y in zip(xs, ys)])


def test_xor_strings():
    xs = codecs.decode("1c0111001f010100061a024b53535009181c", "hex")
    ys = codecs.decode("686974207468652062756c6c277320657965", "hex")
    zs = codecs.decode("746865206b696420646f6e277420706c6179", "hex")
    return zs == xor_strings(xs, ys)

test_xor_strings()
\end{verbatim}

\subsection{Challenge 3 (Single byte XOR decryption)}
\label{sec:org10ffa53}
\begin{itemize}
\item Decrypt a single byte XOR encrypted message
\item The standard frequencies are taken from \href{https://www.wikiwand.com/en/Letter\_frequency\#/Relative\_frequencies\_of\_letters\_in\_the\_English\_language}{here}
\item Similarity measure is the negative of the distance measure taken from \href{https://www.wikiwand.com/en/Bhattacharyya\_distance}{here}
\end{itemize}
\begin{verbatim}
import codecs
import string
from collections import Counter, defaultdict
from math import log, sqrt

n = 1


def compute_ngram_freqs(n=1):
    p = bytes(''.join(open("pride.txt").readlines()).lower(), 'utf-8')
    bis = [p[i:i+n] for i in range(len(p)-n)]
    top = Counter(bis).most_common(500)
    freqs = defaultdict(lambda: 0)
    for b, c in top:
        freqs[b] = float(c) / len(bis)
    return freqs


freqs = compute_ngram_freqs(n)


def english_similarity(m, n=1):
    score = 0
    m = m.lower()
    test_counts = Counter([m[i:i+n] for i in range(len(m)-n)])
    try:
        if (any([codecs.decode(x) not in string.printable
                 for x in test_counts.keys()])):
            return float('-inf')
    except UnicodeDecodeError:
        return float('-inf')
    total_ngrams = sum(test_counts.values())
    for bi in test_counts.keys():
        test_freq = float(test_counts[bi]) / total_ngrams
        score += sqrt(test_freq * freqs[bi])
    if score == 0:
        return float('-inf')
    else:
        return log(score)


def decrypt_single_byte_xor(s):
    plain = b""
    score = float('-inf')
    key = None
    for c in range(256):
        candidate_key = [c] * len(s)
        candidate = xor_strings(candidate_key, s)
        candidate_score = english_similarity(candidate, 1)
        if candidate_score > score:
            score = candidate_score
            plain = candidate
            key = c
    return plain, score, key


def test_decrypt_single_byte_xor():
    s = codecs.decode("1b37373331363f78151b7f2b783431333d78397828372d363c78373e783a393b3736", "hex")
    plain = codecs.encode("Cooking MC's like a pound of bacon")
    return decrypt_single_byte_xor(s)[0] == plain


test_decrypt_single_byte_xor()
\end{verbatim}

\subsection{Challenge 4 (Single byte XOR detections)}
\label{sec:org5d200e5}
\begin{itemize}
\item The file "4.txt" has 60 strings, of which one is single-byte XOR encrypted. We need to find it.
\item Reusing the \texttt{english\_similarity} code from the previous challenge, pick the string with the highest candidate similarity.
\end{itemize}
\begin{verbatim}
import codecs
from operator import itemgetter


def find_single_byte_xor(lines):
    lines = map(decrypt_single_byte_xor, lines)
    return max(lines, key=itemgetter(1))


def test_find_single_byte_xor():
    lines = open("4.txt").readlines()
    lines = map(lambda x: codecs.decode(x.strip(), "hex"), lines)
    return find_single_byte_xor(lines)

print(test_find_single_byte_xor())
\end{verbatim}

\subsection{Challenge 5 (Repeating-key XOR)}
\label{sec:org73734d1}
\begin{itemize}
\item Implement repeating-key XOR
\end{itemize}
\begin{verbatim}
import codecs


def repeating_key_xor(key, plain):
    key = key * (len(plain) // len(key)) + key[:(len(plain) % len(key))]
    return xor_strings(key, plain)


def test_repeating_key_xor():
    plain = codecs.encode('''Burning 'em, if you ain't quick and nimble
I go crazy when I hear a cymbal''')
    key = codecs.encode("ICE")
    cipher = codecs.decode('''0b3637272a2b2e63622c2e69692a23693a2a3c6324202d623d63343c2a26226324272765272a282b2f20430a652e2c652a3124333a653e2b2027630c692b20283165286326302e27282f''', "hex")
    return repeating_key_xor(key, plain) == cipher

test_repeating_key_xor()
\end{verbatim}

\subsection{Challenge 6 (Repeating-key XOR decryption)}
\label{sec:orgc866705}
\begin{itemize}
\item Break repeating-key XOR
\item The file "6.txt" has a string that was encrypted with repeating-key XOR and then base64 encoded.
\item Guess the key-length by computing hamming distances between candidate length sized blocks and testing the candidate lengths with the smallest average hamming distances between blocks.
\begin{itemize}
\item We average hamming distances between 4 blocks, normalize by dividing by the candidate keysize, and then use the five keysizes with the least normalized average hamming distances as candidate keysizes.
\end{itemize}
\item Given the keysize, breaking the ciphertext into blocks and taking the transpose gives you a bunch of single-byte xor problems to solve, the results of which should produce the original key.
\begin{verbatim}
import codecs
from operator import itemgetter
from base64 import b64decode


def hamming(xs, ys):
    return sum(bin(x ^ y).count("1") for x, y in zip(xs, ys))


def test_hamming():
    xs = codecs.encode("this is a test")
    ys = codecs.encode("wokka wokka!!!")
    return hamming(xs, ys) == 37


test_hamming()


def guess_keysize(ciphertext):
    keysize_weights = []
    for n in range(1, 64):
        a, b, c, d = (ciphertext[:n], ciphertext[n:2*n],
                      ciphertext[2*n:3*n], ciphertext[3*n:4*n])
        h = (hamming(a, b) + hamming(b, c) + hamming(c, d) +
             hamming(a, c) + hamming(a, d) + hamming(b, d)) / 6 / n
        keysize_weights.append((n, h))
    return list(map(itemgetter(0),
                    sorted(keysize_weights,
                           key=itemgetter(1))))[0:5]


def transpose(matrix):
    m, n = len(matrix), len(matrix[0])
    return [[matrix[j][i] for j in range(m)] for i in range(n)]


def decrypt_repeated_key_xor():
    ciphertext = b64decode("".join(open("6.txt").readlines()))
    guesses = guess_keysize(ciphertext)
    candidates = []
    for ks in guesses:
        blocks = [ciphertext[i:i+ks] for i in range(0, len(ciphertext), ks)]
        # pad to make transpose faster + easier to write (rectangular matrix)
        blocks[-1] = blocks[-1] + b"\x00" * (ks - len(blocks[-1]))
        trans = map(bytes, transpose(blocks))
        trans_plains = list(map(decrypt_single_byte_xor, trans))
        # make sure all columns get a valid decrypt
        if (any(x[0] == b'' for x in trans_plains)):
            continue

        key = "".join(map(chr, (map(itemgetter(2), trans_plains))))
        trans_plains = list(map(itemgetter(0), trans_plains))
        plaintext = map(bytes, transpose(trans_plains))
        plaintext = b"".join(plaintext)
        candidates.append((plaintext, key))
    return max(candidates, key=lambda x: english_similarity(x[0]))


print(decrypt_repeated_key_xor()[1])
\end{verbatim}
\end{itemize}

\subsubsection{Mistakes}
\label{sec:org6725b33}
\begin{itemize}
\item Messed up splitting into blocks
\item Messed up transposing multiple times
\end{itemize}
\subsection{Challenge 7 (AES ECB decryption)}
\label{sec:org4966adf}
\begin{itemize}
\item Decrypt a file in AES ECB mode
\end{itemize}
\begin{verbatim}
def aes_ecb_decrypt(key, ciphertext):
    pass
\end{verbatim}
\subsection{Challenge 8 (Detect AES in ECB mode)}
\label{sec:org8120f9d}
\begin{itemize}
\item Detect AES in ECB mode
\item Look for duplicate 16 byte blocks
\end{itemize}
\begin{verbatim}
def is_probably_aes_ecb(ciphertext):
    blocks = [ciphertext[i:i+16] for i in range(0, len(ciphertext), 16)]
    return len(blocks) != len(set(blocks))
\end{verbatim}
\section{Set 2}
\label{sec:org56570f3}
\end{document}